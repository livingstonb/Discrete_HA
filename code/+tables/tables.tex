\documentclass{article}

\usepackage{booktabs}
\usepackage{amsmath}
\usepackage[tableposition=top]{caption}
\usepackage{geometry}
\usepackage{pdflscape}
\usepackage{threeparttable}

\addtolength{\topmargin}{-0.75in}
\addtolength{\textheight}{1.5in}

\begin{document}

\begin{table}
\centering
\caption{Baselines}
\input{table1.tex}
\end{table}

\begin{table}
\centering
\caption{Baseline Decompositions}
\input{table2.tex}
\end{table}

\clearpage

\begin{table}[h]
\caption{Alternative Asset Targets}
\begin{threeparttable}
\centering
\input{table3.tex}

\begin{tablenotes}
	\item[*] Where indicated, households are classified as HtM when $a \leq 1\%$ mean annual income.
\end{tablenotes}
\end{threeparttable}
\end{table}

\begin{table}[h]
\caption{Bequests, Death, and Annuities}
\begin{threeparttable}
\centering
\input{table4.tex}

\begin{tablenotes}
	\item[*] Where indicated, households are classified as HtM when $a \leq 1\%$ mean annual income.
\end{tablenotes}
\end{threeparttable}
\end{table}

\begin{table}[h]
\caption{Discount Factor Heterogeneity}
\begin{threeparttable}
\centering
\input{table5.tex}

\begin{tablenotes}
	\item[*] Where indicated, households are classified as HtM when $a \leq 1\%$ mean annual income.
\end{tablenotes}
\end{threeparttable}
\end{table}



\end{document}